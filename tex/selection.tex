\chapter[Object and Event Selection][Object and Event Selection]{Object and Event Selection}
\label{chapter:selection}

As stated in Chapter \ref{chapter:analysis}, the analysis is divided into 3 channels based
on lepton counting: 2 same-charge leptons, 3 leptons and 4 leptons. The lepton
counting occurs for fully identified leptons with full overlap removal with
transverse momentum over 10 \gev to ensure orthoganality. Lepton selections are tightened
afterward within each channel.

The cuts for each channel are provided in Table \ref{} and the object selections are detailed in the
following selections. The selections are based on optimizations of the channel sensitivity
performed using MC (event for data driven backgrounds) and adhoc values for normalization systematic errors. The optimziation is
detailed in Section \ref{section:Optimization}.All channels are comprised of three basic requirements: the presence of b-tagged
jets, the presence of additional light jets, and a veto of same flavor opposite sign leptons with an
invariant mass within the Z window. Additional requirements on the invariant mass of the leptons, the missing transverse energy
in the event, and the total object energy ($\rm H_{T}$) proved to have negligble additional benefit at our level of 
statistics.


\section{2l SS Channel}

\section{3l Channel}

\section{4l Channel}

\section{Electron Selection}

The electrons are reconstructed by a standard algorithm of the
experiment~\cite{EgammaSF} and the electron cluster is required to be fiducial 
to the barrel or endcap calorimeters: $|\eta_{cluster}| < $ 2.47. Electrons
in the transition region, $1.37 < |\eta_{cluster}| < 1.52$, are vetoed.
Electron reconstruction and identification is discussed in depth in Chapter \ref{chapter:electron}
Electrons must pass the the \textsc{VeryTight} likelihood identification criteria.

In order to reject jets (b-quark jets in particular) misidentified as electrons,
electron candidated  must also be well isolated from additional tracks and
calorimeter energy around the electron cluster. Both the tracking 
and calorimeter energy within $\Delta R=0.2$ of the electron
cluster must be less than 10\% of the electron transverse momentum: $ptcone20/P_t <$ 0.1 and $Etcone20/E_T <$ 0.1.
All quality tracks with momentum greater than 400 MeV contribute to the isolation
energy.  Calorimeter isolation energy is calculated
using topological clusters with corrections for energy leaked from the
electron cluster~\cite{Topo}. Pile-up and underlying event corrections are applied using
a median ambient energy density correction, developed in~\cite{PileupCorrections}. 

The electron track must also match the primary vertex. The longitudinal projection 
of the track along the beam line, $z0\sin{\theta}$, must be less than 1 cm) and the transverse projection divided by the
parameter error, $d0$ significance,must be less than 4. These cuts are used in particular to suppress backgrounds
from conversions, heavy-flavor jets and electron charge-misidentifications. 


The electron selection is provided in Table~\ref{selection:table_object}. 


\section{Muon Selection}

Muons used in the analysis are formed by matching reconstructed inner detector
tracks with either a complete track or a track-segment reconstructed in the muon spectrometer (MS).
The muons must satisify $\abs{\eta} < 2.5$.
The muon track are required to be a good quality combined fit of inner detector hits and muon
spectrometer segments, unless the muon is not fiducial to the
inner detector, $\abs{\eta > 2.4}$.  Muons with inner detector tracks are further required
to pass standard inner detector track hit requirements~\cite{MCP2012}.  

As with electrons, muons are required to be isolated from 
additional tracking or calorimeter energy: $ptcone20/P_t <$ 0.1, $Etcone20/E_T <$ 0.1) A cell-based $Etcone20/P_T$ relative
isolation variable is used. A pile-up energy subtraction based 
on the number of reconstructed verticies in the event is applied. The
subtraction is derived from a Z boson control sample.


The muons must also originate from the primary vertex and have impact parameter requirements, $d0$ significance $<$ 3, and $z0\sin{\theta} <$ 0.1 cm, similar to the electrons. 


The muon selection is provided in Table~\ref{selection:table_object}. 

\section{Jet and b-Tagged Jet Selection}

Jets are reconstructed in the calorimeter using the anti-$k_t$~\cite{Cacciari:2008gp} algorithm
with a distance parameter of 0.4 using locally calibrated
topologically clusters as input (LC Jets). 

Events with any LooserBad jet are vetoed. Jets near a hot Tile cell in data periods
B1/B2 are rejected. The local hadronic calibration is used for
the jet energy scale, and ambient energy corrections are applied to account
for energy due to pileup.

\pt~ and $\eta$~ cuts are tuned based on the sensitivity to \tth
as explained in section~\ref{sec:optim-jets}. 

For jets within $|\eta| < 2.4$ and $p_T <$ 50~GeV, are required to be
associated with the primary vertex, the ``jet vertex fraction'' (or JVF),
which is the fraction of track $p_T$ associated with the jet that comes from the primary vertex,
must exceed 0.5 (or there must be no track associated to the jet). 

B-jets are tagged using a Multi-Variate Analysis (MVA) method called MV1 and relying on information
of the impact parameter and the reconstruction of the displaced vertex of the
hadron decay inside the jet.%~\cite{ATLAS-CONF-2011-102}.
 The output of the tagger is required to be above 0.8119 which corresponds to a $70\%$ Working Point (WP).



\section{Optimization}

\label{section:optimization}

