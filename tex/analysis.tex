%% \chapter[htoc-titlei][hhead-titlei]{htitlei}
%% -----------------------------------------------------------------------------
\chapter[Search for the TTH Decay in the Multilepton Channel][Search for the TTH Decay in the Multilepton Channel]{Search for the TTH Decay in the Multilepton Channel}
\label{chapter:analysis} 
This chapter provides an overivew of the the set of analyses searching for the Standard Model
(SM) production of the Higgs boson in association with top quarks in
multi-lepton final states with multiple jets (including b-quark tagged jets).
Searches in $t\bar{t}H$ final states with 2 same-charge, 3 and 4 light leptons
($e, \mu$) are discussed in depth. These final states target specifically Higgs
decays to vector bosons, $H\rightarrow W^{\pm}W^{\pm}$ and $H\rightarrow
Z^{\pm}Z^{\pm}$ and form a complement to searches for $t\bar{t}H$ production in
final states targeting the $H\rightarrow b\bar{b}$ \cite{paper1},
      $H\rightarrow\gamma\gamma$\cite{paper2}, and $H\rightarrow\tau\tau$ decay modes.


Based on SM production cross-sections, observation lies just outside the sensitivity
of the Run I dataset, even when combining all searches. The analyses provide an opportunity to 
constrain for the first time the \tth production mode with limits reasonably close to the
actual production rate. As such the analysis is optimized to overall sensitivty to the 
\tth production rather than individual decay modes, which would be more useful for
constraining Higgs couplings. 


Detailed description of the event and objection section are provided in Chapter \ref{chapter:selection},
background modelling in Chapter \ref{chapter:background}, the effect of syetmatic errors and the 
statistical analysis in Chapter \ref{chapter:systematics} and final results in Chapter \ref{chapter:results}.


\section{Signal Characteristics} 
\tth can be observed in a number of different final states realted to the
the Higgs boson and the top quark decay modes.

Three Higgs boson decays are relevant for this analysis: \WW,
\twotau and \ZZ. The top and anti-top quarks decay  in
\Wb. Each \W boson decays either 
leptonically (l=$e^\pm$, $\mu^\pm$,$\tau^\pm$) with missing energy or hadronically. 
Table \ref{ana:table_decay} provides the fractional contribution of the main 
Higgs decay modes at the generator level to \tth search channels. These
numbers will be modified by lepton acceptances. 

\begin{table}[htbp]
  \begin{center} 
    \caption{Contributions of the main Higgs decay modes to the 3 multilepton
      \tth signatures at generation level.
      }\label{ana:table_decay} {\small
    \begin{tabular}{l|c|c|c} 
      \hline\hline
  Signature & $H \rightarrow WW$  & $H\rightarrow \tau\tau$  & $H \rightarrow
  ZZ$  \\\hline
  Same-sign &  $100\%$ & -- & -- \\
  3 leptons  &  $71\%$ & $20\%$ & $9\%$ \\
  4 leptons  &  $53\%$ & $30\%$ & $17\%$  \\
     \hline
    \end{tabular}}
  \end{center}
\end{table}


All modes are generally dominated by the $WW$ signature, though the 3l and 4l
channnels possess some contribution from the$\tau\tau$ and $ZZ$ decays. 


The signal is expected to be characterized by the presence of 2 b-quark jets from
the top quark decays, leptons from vector boson and tau decays,
a high jet multiplicity, and missing energy. In general, the number of leptons is anti-correlated 
with the number of jets, since a vector boson can either decay leptonically 
or hadronically. For \hww, the light quark multiplicity, $N_q$, and the
number of leptons, $N_l$, follow this relation: $2N_l+N_q+N_b=10$.

\begin{itemize}
\item In the same-sign channel, the \tth final state contains 6 quarks. These events
are then characterised by a large jet multiplicity.

\item In the 3 lepton channel, the \tth final state contains 4 quarks from the hard scatter.

\item In the 4 lepton channel, the \tth final state contains a small number of light
quarks, 0 (\hww case), 2 or 4 (\hzz case).


\end{itemize} 

\section{Background Overview}

Background processes can be sorted into two categories:

\begin{itemize}

\item Events with a non prompt or a fake lepton selected as prompt
  lepton. These processes cannot lead to a final state compatible with the
  signal signature without a misreconstructed object. This category includes
  events with a prompt lepton but with misreconstructed charge and events
  with jets that "fake" leptons. 

  The main backgrounds of this sort are: \ttbar and \zj.
  Data-driven techniques are used to control some of these processes.
  Their importance varies depending on the channel.

\item Events which can lead to the same final state as the signal (irreducible
  backgrounds).
 The main background of this category are: \ttV, \WZ, and \ZZ.
 They are modeled using the Monte Carlo simulations.

\end{itemize}


\section{Analysis Strategy} 


The analysis search is conducted in 3 channels, based on counting of fully identified
leptons: 2 SS leptons, 3 leptons,and 4 leptons. The lepton counting occurs before additional object cuts are made in each individual 
channel to ensure orthoganility. The division into lepton channels rather than channels targeting specific decay modes
allows channels with different sensitivities to considered separately. We further divide the 2l SS into sub channels
based on the number of jets and flavor of the leptons and the 4l channel into subchannels enriched and depleted in OS leptons arrising from Z decays. 

The channels are fed into a posson model 



