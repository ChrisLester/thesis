%% \chapter[htoc-titlei][hhead-titlei]{htitlei}
%% -----------------------------------------------------------------------------
\chapter[Electrons][Electrons]{Electrons}

This chapter details the contributions I made to electron identification and efficiency measurments. It is not essential to continuity of the thesis in general but provides in depth documentation of the work I completed for the experiment. I focus on the electorn identification

\section{Electrons at Hadron Colliders}

High energy electron signatures are important elements of seaches and measurements at hadron colliders. The overwhemling majority of collisions that deposit energy in the detectors are the result of strong-force mediate interactions of the constituent partons. These collisions result in the production of high energy jets in the detector. Figure XX shows the cross-sections of various processes as a function of the center of mass energy of the collision. Physics invovling the electroweak interaction or even strong production of massive states occur many orders of magnitdue less frequently than the total inelastic cross-section. 
Interesting physics signatures, both stardard model and beyond, often involve the production of light leptons as a result of the decay of massive particles. Choosing events that have high energy electrons or muons targets events that contain electroweak vertices and dramatically reduce the background from the more copiously produced strong physics. Electron and muon energy and momenta are also relatively well-measured compared to jets. This allows for the use of well-resovled kinematic shapes used discriminate the sigantures of different processes in analyses. 

At ATLAS, the primary datasets for most analyses are collected with electron and muon triggers. Electron triggers are particularly important, because the muon triggers system has a 20\% small acceptance than the electrons. The challenge in identifying electrons is distguising the production of electrons from direct production of W and Z decays from electrons produced in the more copiuously produced b-meson decays, fake-electron signatures from rare jet fragmentations into charged and neutral pions, and photon conversions in the inner detector. The identification of electrons, the precise measurement of the identification efficien, and the measurement of the rate of fake electron signatures lead are often the most important and challengin peices of an analyis. The following sections discuss the identification of electrons for the primary electron trigger and offline physics analyses as well as the measurement of the electron identification efficiency in 2012. Because I had a major role in these projects, I will at times discuss their historical evolution and not simply focus of the particular measurement relevant to the \tth analysis.  


%\section{Reconstruction of Electron at ATLAS}
%
%High-energy electrons are charged particles that produced both a track in the inner detector and cluster of energy in the calorimeter. The electron reconstruction algorithm is discussed in detail in XXXX. Electrons are seeded by their clusters of energy in the liquid argon calorimeter (LAr). A sliding window algorithm uses rectangular regions in $\phi$ and $\eta$ of a fixed number of calorimeter cells and translates in both directions to maximize the amount of energy in window. For the barrel and endcap calorimeters ($|\$eta| < 2.47$, the windows are 3x7 and 5x5 cles respectively ($\eta$x$\phi$). They are wider in phi to allow for energy smearing in phi due to bremstrahlung and the curvature of electron track in the magnetic field. The exact cell sizes are give in section XXXX. Windows containing at least $E_T < 2.5$ GeV of energy in the second layer of the calorimeter. 
%
%To distinguish from photons, which undergo the same seeding algorithm (albiet with a different window size), electron energy cluster must match to a reconstructed track within a window of 0.05×0.10 ($\eta$x$\phi$) and with loosely matching calorimeter energy and track momentum $E/p$ < 10. Photons, however, often convern in the inner detector material. To resolve the ambiguity, electrons that match to tracks that are already identified as part of a conversion vertex during the previously run conversion re-construction are removed from the electron candidate container. 
%
%Material in front of the calorimeter makes electron reconstruction difficult due to electron energy loss through bremmstrahlung. Standard track fitting is accomplished via a least squares fit to linearlized helical model with accounting for the accumlated scattering of pion or muon-like particle. This sort of fitting fails to adqueatly describe the change in curvature from abrupt energy losses during bremmstrahlung that occur for the much lighter electron. This results in non-insigificant inefficiencies of matching tracks to electron clusters when the standard tracking algorithms are used for electron candidates, especially lower energy electron candidates, whose tracks bend more in the magnetic field. Moreover, as shown in Figure XXXX the material profile of the inner detector has a strong eta dependence due to the TRT and SCT services. This results in large eta dependencies in the resolution of electron track parameters.
%
%For reconstruction of data collected in 2012, a new fitting techique was introduced that incorporataed bremstrahlung energy losses through a Gassiun Sum Filter (GSF) XXXX. Figures XXXXa and XXXXb show the improvement in the electron reconstruction efficiency from 2011 to 2012, achieved through the introduction of this filter and the improvement in a the track impact parameter resolution for electron tracks. Because the fitting is cpu intensive, it is not run online during data collection.  
%
%Electrons are recalculated %Energy cluster, 4 vectors%
%%%%% Material plots 
%%%%% GSF plots 

\section{Identification of Electrons at ATLAS}

Electron reconstruction 


\subsection{Pile-up and Electron identifiation}

Plots of pile up differences in distributions 

\subsection{2011 Menu and Trigger}

\subsection{2012 Menu and Trigger}

\subsection{Electron Likelihood}

\section{Measurement of Electron Efficiency at ATLAS}

\subsection{Techniques}

\subsection{Issues}

