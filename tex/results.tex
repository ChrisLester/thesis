\chapter[Results and Statistical Model][Results and Statistical Model]{Results and Statistical Model} 
\label{chapter:results} 


\section{Results in Signal Regions}



\section{Statistical Model}
NEED TO DEFINE MU IN ANALYSIS SUMMARY CHAPTER


We use the above results to make two sets of measurements: an upper confidence limit on $\mu$, the signal strength parameter, and a measurement of $\mu$. These measurements are done for each channel individually and then combined. 
The interpretation of the results in the form of a statistical model follow the procedure, discussed here \cite{asym}. We interpret the results as counting experiments in each signal region. Therefore agreement in kinematic shapes do not affect the statistical procedure. 




\subsection{The Likelihood} 
The observed and expected event yields in the signal regions are analyzed using a binned likelihood function ($\mathcal{L}$), built from product of Poisson models of expected event counts for each bin, where the bins for our case are the separate signal regions:
\begin{equation}
\mathcal{L}\ \alpha\  \prod_{i=0}^{N_{ch}} P(N_{obs}^{i}| \mu \cdot s_{exp}^{i} + b_{exp}^{i})
\end{equation}
where $s_{exp}^i$ is the SM signal expectation in the signal region, $b_{exp}^i$ are the background expectations, $i$ counts over the signal regions, and P is the Poisson distribution. The signal strength parameter is the parameter of interest in the model (POI) and acts as a simple scale-factor to the SM \tth production rate and is common to all channels. Setting $\mu$ to 0 corresponds to the background only scenario. The background parameter, $b$, is a sum over all background processes. 

The signal and background expectations ,$s$ and $b$, depend on systematic errors. These are included in the likelihood function in the form of a vector nuisance parameters, $\vec{\theta}$, which are constrained to fluctuate within Gaussian distributions. These fluctuations affect the background and signal expectations by response functions, $\nu(\vec{\theta})$, set by uncertainties measured in the previous section. For instance, the \WZ normalization uncertainty is 50\% from Section~\ref{section:wz} and is included in the fit as its own unit gaussian,$G(\theta|0,1)$. The fluctuations of the gaussian, $\theta_{WZ}$ scale the background contribution via the form, $0.5\cdot(1+\theta_{WZ})\cdot b_{WZ}$. For many of the detector systematics, the uncertainties are two sided and are included as piecewise Gaussians. We add correlations to various uncertainties by hand, when appropriate. With these nuisance parameters, the likelihood takes this form:
\begin{equation}
\mathcal{L}(\mu,\vec{\theta})\ =  \left( \prod_{i=0}^{N{ch}} P(N_{obs}^{i}; \mu \cdot \nu_{s}(\vec{\theta})\cdot s_{exp}^{i} + \nu_{b}(\vec{\theta})\cdot b_{exp}^{i}) \right) \times \prod_{j}^{N_{\theta}}G(\theta_j; 1,0)
\end{equation}


\subsection{Test Statistic and Profile Likelihood}

Values of $\mu$ are tested with the negative log quantity, $q_{\mu}= -2$ln$(\lambda(\mu))$, where $\lambda(\mu)$ is the test statistic.
$\lambda(\mu)$ is defined as:
\begin{equation}
\lambda(\mu) \equiv \frac{\mathcal{L}(\mu,\hat{\vec{\theta_{\mu}}})}{\mathcal{L}(\hat{\mu},\hat{\vec{\theta}})}
\end{equation}
where $\hat{\vec{\theta_{\mu}}}$ are values of the nuisance parameter vector that maximize the likelihood for a given value of $\mu$ and $\hat{\mu}$ and $\hat{\vec{\theta}}$ are the fitted values of signal strength and nuisance parameters that maximize the likelihood overall. $\mu$ is constrained to be positive.  

\subsection{ CL$_{s}$ Method}

Exclusions limits on the signal strength are calculated with the test statistic using a modified frequentist method, called the CL$_{s}$ method\cite{0954-3899-28-10-313}. CL$_{s}$ is defined as a ratio of two frequentist quantities. The numerator quantifies the probability of finding the observed data given the signal $+$ background hypothesis. The denominator quantifies the probability of the data given the background only hypothesis.

Using the numerator alone has the undesirable property that, if the data fluctuates below the expectation, an exclusion limit can be reached that is far beyond the sensitivity of the experiment. Normalizing to the background only hypothesis penalizes these low sensitivity cases.

The probability of obtaining an observation as extreme as the data given a particular signal $+$ background hypothesis is given by the p-value,$p_{\mu}$ defined as:
\begin{equation}
 p_{\mu} = \int_{q_{\mu}^{obs}}^{\infty} f(q_{\mu}) dq_{\mu}
\end{equation}
and the probability of obtaining an observation as extreme as the data given the background hypothesis, $p_b$ is:
\begin{equation}
 p_{b} = \int_{q_{\mu=0}^{obs}}^{\infty} f(q_{\mu=0}) dq_{\mu=0}
\end{equation}
where $f(q_{\mu})$ is the distribution of $q_{\mu}$ for all possible observations for a given $\mu$ and $q$ is defined above. Therefore,
\begin{equation}
 CL_{s} = \frac{p_{\mu}}{1-p_b}
\end{equation}
. A $\mu$ value is considered excluded at 95\% confidence when CL$_{s}$ is less than 5\%. 

\subsection{Exclusion Limits}

Table \ref{} shows {\textit expected} exclusion limits for all channels, including the analysis uncertainties cumulatively. It shows the relative importance of the statistical and systematic uncertainties to the analysis sensitivity. The {\textit observed} limits using observed data and predictions can be seen in Figures~\ref{}-\ref{} for splitting and combining the sub-channels and in Table XX by numbers. We expect a combined limit of XXX (background only) and XXX (signal injected) and see a limit of XXX. The channel sensitivity is dominated by the 2$\ell$ and 3$\ell$ channels.

\begin{table}[htbp]
\begin{center}
\begin{tabular}{|c|c|c|c|c|c|}
\hline 
\multicolumn{2}{|c|}{ Channels} &  Stat &  +Fakes Unc.  & +Theory  & + Experimental\\ 
\hline 
2$\ell$       & 2$\ell$ee  & 7.44 & 8.52 & 8.82 &8.94 \\ 
         & 2$\ell$em   &3.46 & 3.81 & 4.07 &4.18 \\ 
        & 2$\ell$mm   & 4.03 & 4.14 & 4.47 &4.57 \\ 
        & 2$\ell$tau   &8.08 & 8.92 & 10.00 &10.03 \\ 
        &  All   & 2.16 & 2.44 & 2.81 &2.90 \\ 
\hline 
\multicolumn{2}{|c|}{ 3$\ell$ }  &3.40 & 3.43 & 3.59 &3.66 \\ 
\hline 
\multicolumn{2}{|c|}{ 4$\ell$ }  & 15.16 & 15.16 & 15.44 &15.55 \\ 
\hline 
\multicolumn{2}{|c|}{ 1l2tau }  & 10.41 & 13.84 & 14.20 &14.22 \\ 
\hline 
\multicolumn{2}{|c|}{ All } & 1.68 & 1.85 & 2.14 &2.22 \\ 
\hline 
\end{tabular} 
\caption{\label{table:results_cumulative} 95$\%$CL limits on $\mu$ for all channels and combination with cumulative uncertainties.}
\end{center} 
\end{table} 

%\begin{figure}[htbp]
%   \centering
%      \includegraphics[width=0.48\textwidth]{figs/results/Limits} 
%      \includegraphics[width=0.48\textwidth]{figs/results/Limits_simple} 
%  \caption{Summary of limits results per channel.}
%  \label{figure:results_limits}
% \end{figure}


\subsection{$\mu$ Measurements}

In addition to setting a limit on the signal strength, we also fit the best value of the signal strength for $\mu$. We do this by minimizing the negative log likelihood value, $q_{\mu}$ or conversely maximizing the likelihood. The 1-$\sigma$ error band is set via a profile likelihood scan, where the value $q_{\mu}$ is scanned as a function of $\mu$. Values of $\mu$ that increase $q^min_{\mu}$ by 1 form the edges of the error band. The fitted values of $\mu$ with errors are provided in Table XXX for each sub-channel fit as well as the combined fit. 

\subsection{Nuisance Parameter Impact on the Signal Strength}

Finally, we examine the post-fit impact of the various nuisance parameters on the final fit. We expect to have measured the various analysis uncertainties well and do not expect the fit to have much further constraint. For that reason, we expect the pulls of the nuisance parameters to be close to 0 and the measured uncertainties on those parameters to be consistent with the input uncertainties. 
Figures XXXX show. 



