\newpage
\section{Objects definitions}
\label{sec:objects}

The identification of electrons, muons and jets as well as the b-tagging
algorithm are summarised in this section. The various properties of theses
objects are tuned in the optimisation section~\ref{sec:optimisation}

%Following statements are larely inaccurate. No explicit Nvtx cut made, main role of
%vertex counting now is to apply various pileup corrections - PO
%\subsection{Primary vertex}
%% keep here ?
%Events are required to have at least one vertex from at least five tracks
%with $p_T>0.4~GeV$; among all vertices that satisfy this criterion, the
%primary vertex is defined as the one with the highest summed track $p_T^2$.

\subsection{Jet definition and b-tagging}
\label{sec:jets}
Jets are reconstructed in the calorimeter using the anti-$k_t$~\cite{Cacciari:2008gp} algorithm
with a distance parameter of 0.4 using locally calibrated
topologically clusters as input (LC Jets). Events with any LooserBad jet are vetoed. Jets near a hot Tile cell in data periods
B1/B2 are rejected. The local hadronic calibration is used for
the jet energy scale, and ambient energy corrections are applied to account
for energy due to pileup.

%  Jets are required to satisfy $p_T > 25$~GeV and
%$|\eta|<2.5$.
% These two conditions were tuned based on the sensitivity to \tth
%as explained in section~\ref{sec:optim-jets}.

\pt~ and $\eta$~ cuts are tuned based on the sensitivity to \tth
as explained in section~\ref{sec:optim-jets}. 
 For jets within $|\eta| < 2.4$ and $p_T <$ 50~GeV, are required to be
associated with the primary vertex, the ``jet vertex fraction'' (or JVF),
which is the fraction
of track $p_T$ associated with the jet that comes from the primary vertex,
must exceed 0.5 (or there must be no track associated to the jet). 

B-jets are tagged using a Multi-Variate Analysis (MVA) method called MV1 and relying on information
of the impact parameter and the reconstruction of the displaced vertex of the
hadron decay inside the jet.%~\cite{ATLAS-CONF-2011-102}.
 The output of the tagger is 
required to be above 0.8119 which corresponds to a $70\%$ Working Point (WP).

%%{\bf NB Move to MV1c planned; also the WP is an EM point, not LC point}

\subsection{Electron definitions}
\label{subsec:electrondescription}

The electrons are reconstructed by a standard algorithm of the
experiment~\cite{EgammaSF} and the electron cluster is required to be fiducial 
to the barrel or endcap calorimeters: $|\eta_{cluster}| < $ 2.47. Electrons
in the transition region, $1.37 < |\eta_{cluster}| < 1.52$, are vetoed. 

Each analysis channel tested different momentum and identification selections.

After optimisation, all analyses will use the VeryTight Likelihood Identification operating
point. The electron likelihood is well documented and trigger and offline data-to-MC
efficiency scale-factors are provided by the e-gamma performance
group~\cite{EgammaSF}. 

On top of the standard identification, additional relative calorimeter and
tracking isolation as well as track impact parameter cuts  are applied. The tracking
isolation variable ($PtCone20/P_T$) used is common among all analyses. All
quality tracks within a cone size of $\Delta R=0.2$ around the electron
candidate with momentum greater than 400 MeV contribute to the isolation
energy.  The relative calorimeter isolation $EtCone20/P_T$ variable used is
also common among all analyses and is based on the e\-gamma recommended method
using topological clusters with corrections for energy leaked from the
electron cluster~\cite{Topo}. Pile-up and underlying event corrections are applied using
a median ambient energy density correction, developed in~\cite{PileupCorrections}. 

The impact parameter cuts used the longitudinal projection along 
the beam line ($z0\sin{\theta}$) and the transverse projection divided by the
parameter error ($d0$ significance). These cuts are used in particular to suppress
the heavy flavor and conversion backgrounds. 

The list of electron cuts for each analysis are provided in Table~\ref{tab:obj-final}. 

%ATL-COM-PHYS-2011-1186 - corretions

\subsection{Muon definitions}
\label{subsec:muondescription}

Muons used in the multi-lepton analysis are reconstructed using the Muid (Chain 2)
family of algorithms, which include muons from both Muid or MuGirl algorithms.
The ``tight'' identification working point is used, which effectively requires
the muon tracks to be a combined fit of inner detector hits and muon
spectrometer segments.  Muons are further required
to pass standard inner detector track hit requirements. These
selection requirements are provided by the Muon Combined Performance group which
provide the data-to-MC efficiency scale-factors~\cite{MCP2012}.  
Performance of the analyses were also tested with ``Combined'' muons.

Relative Tracking and calorimeter isolation as well as track impact parameter
cuts are also applied. The tracking isolation variable used is similar to the one
used by the electrons and discussed above ($PtCone20/P_t$). However, two different
calorimeter isolations are used. A cell-based $EtCone20/P_T$ relative
isolation variable is used. A pile-up energy subtraction based 
on the number of reconstructed verticies in the event is applied. The
subtraction is derived from a Z tag and probe sample and was originally
developed for the $H\rightarrow W^{+} W^{-}$ search~\cite{HSG3Objects}. 

The isolation of charged leptons is often defined using the transverse energy found in a fixed cone
around the lepton position. Because the angular distance between the charged lepton and the b-
quark decreases as the top quark Lorentz boost increases, a smarter definition of isolation is beneficial. Developed in the context of boosted top quark searches, it has been shown to be of help also in the resolved regime for top quark measurements.
For this reason, it is considered also in this analysis
The mini-isolation is defined as: 
\begin{equation}
I^{\mu}_{mini}=\Sigma_{tracks} p_T^{track}
\end{equation}
where the sum runs over all tracks (except the matched lepton track) that have $p_T^{track} >$ 1 GeV, pass quality cuts and have $\Delta R(\mu, track) < k_T /p_T^\mu$. Here $p_T^\mu$ is the muon transverse momentum and $k_T$ is an
empirical scale parameter (10 GeV) optimised for multijet background rejection. The optimisation is run on the relative mini-isolation $I^{\mu}_{mini}/p_T^\mu$. 

As for
electrons, impact parameter, $d0$ significance, and $z0\sin{\theta}$ cuts are
also applied. 

The optimisation documented below also consider other calorimeter isolation variables,
which used the above calorimeter isolation variable with cuts tuned in $\eta$ 
and $E_T$ bins to be 90\% efficent and the mini-isolation variables, developed
by the top group to have a shrinking cone-size with object \pt \cite{MiniIso}.  


\subsection{Hadronic tau definitions}
\label{subsec:hadtaudescription}
Taus decaying hadronically are reconstructed using clusters in both the electromagnetic and the 
hadronic calorimeters. Hadronic tau candidates are seeded with the anti-$k_t$ algorithm
described in the jet section. They are required to have a transverse momentum $p_T>25$~GeV and a
pseudo-rapidity $|\eta|<$2.47.
Hadronic taus are identified using %the {\it loose}
a working point of a Boosted Decision Tree (BDT) method ({\it loose}, {\it
medium} or {\it tight}).
%Although, results from the optimisation show that the {\it loose} identification gives a better sensitivity to the leptons $+$ tau channel
Hadronic tau candidates are further asked to have a charge $\pm1$ and to have either one or three prongs.
Muon and electron veto, in order to reject leptons faking taus can be applied. The electron veto is a BDT with three possible working points: {\it loose}, {\it medium} and {\it tight}. In addition to this veto, an
overlap removal algorithm requires the removal any tau overlapping with an
electron or a muon in a cone of size $\Delta R < 0.1$
and to remove jets overlapping with taus in a cone of size $\Delta R < 0.3$.

\subsection{Overlap Removal}
\label{subsec:hadtaudescription}
Overlap removal is applied to all fully identified objects, discussed above, based on overlapping cones in $\Delta R$.
The order and $\Delta R$ cut is given in Table~\ref{tab:overlap}. $\mu$-jet
overlap removal cut is tuned in Section~\ref{sec:optimisation}.%%.

\begin{table}[htbp]
  \begin{center} \label{tab:overlap}
    \caption{ Object Overlap Removal Cuts 
      } {\small
    \begin{tabular}{c|c|c} 
      \hline\hline
  Static Object & Removed Object  & $\Delta R$ Cut  \\
  \hline
  Muon &  Electron  & $0.1$  \\
  Electron &  Jet & $0.3$  \\
  Jet  &  Muon & $0.3$, $0.4$, $0.04 + \frac{10 GeV}{p_{T}}$  \\
  Electron &  Tau  & $0.2$  \\
  Muon &  Tau & $0.2$  \\
  Tau  &  Jet & $0.3$  \\

     \hline
    \end{tabular}}
  \end{center}
\end{table}



