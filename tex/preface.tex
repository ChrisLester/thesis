chapter*{Preface}
\addcontentsline{toc}{chapter}{Preface}

This analysis represents only the last year of work of my graduate career and was accomplished as part of
a $\sim10$ person team. I was involved in some way at every level 
of the analysis, but for some parts I had a more important role.
I led the team that managed the datasets and analysis coding infrastructure, developed the techniques
for estimating the fake and vector-boson backgrounds and optimized
the selection for the 3 lepton signal region. I also directly supervised the work of multiple students, 
each of which performed a critical task for the analysis: measurement of the isolation
selection efficiency for leptons, assessment of the impact of the experimental
systematic uncertainties on the analysis and development of the 4 lepton signal region. 

Earlier in my graduate career, I focused primarily on performance work for the ATLAS experiment.
In 2010 I helped create an in-situ mapping of the inner detector material using photon conversions.  
As a part of small Penn-led team, I developed the electron
identification algorithms that formed the basis of the primary electron trigger for the
2011 and 2012 runs and are used by many ATLAS analyses. Electrons are critical 
pieces of analyses at hadron colliders but also are challenging to understand
due to large and diverse backgrounds. Our team oversaw the development of electron identification
from very basic cut-based roots to the extremely robust and high performing
multi-variate algorithms used in this thesis. We managed this at a time
when the running conditions were changed rapidly. Later, I pushed for the proper
measurement of electron identification efficiency at lower energies that 
were important for finding Higgs signals in a number of channels.

Before the \tth\ analysis, I worked on two additional analysis projects. The first
was a search for the Higgs in the \WW decay channel. For this analysis
I played a small role as the coordinator for non-\WW di-boson background measurements.
The second was a search I explored for the Higgs in the $W^{\pm}H$ production
mode with same-sign dileptons. As many things in science, this ended up
being a dead-end: a difficult analysis without compelling sensitivity. 

I had the privilege and luck of being part of some truly amazing research at a very important
time in the field. While experimental particle physics is a collaborative effort,
I can look at my small set of accomplishments with pride in the fact that they helped 
contribute to the overall success of the experiment and to push our fundamental 
understanding of the universe the tiniest bit forward. 


\vspace{0.05\textheight}

\begin{tabular}{p{0.5\textwidth} l}
  & Chris Lester            \\
  & CERN, Fall 2014   \\
\end{tabular}

