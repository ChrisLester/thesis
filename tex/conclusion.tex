%% \chapter[htoc-titlei][hhead-titlei]{htitlei}
%% -----------------------------------------------------------------------------
\chapter[Conclusions][Conclusions]{Conclusions}

This thesis has presented a preliminary search for the \tth\ process in multilepton
final states using the 2012 8 TeV dataset, collected by ATLAS. The signal regions,
which are binned in number of leptons, show an interesting excess of events. The level of excess is not large enough
to claim statistically signficant observation of a new process, whether SM \tth\ production or otherwise. As
a result, we proceed by setting a 95\% confidence limit on the \tth\ production rate
compared to the SM production,which is much less strict than expected and provide a fitted value of $\mu$,
the ratio of the observed production rate to the theoretical SM \tth\ production rate.

We believe the background processes are well-measured via simple and 
straightforward procedures. These procedures either rely on or are verified by
data control and validation regions surrounding the signal regions. The good
data-MC agreement in these regions suggests that the data excesses are characteristic
of the signal regions only and that the background processes are well-understood
within the systematic uncertainties assigned. 

The signal regions are constructed with simple selection criteria, requiring 
only certain number of leptons, jets, and b-tagged jets within standard
acceptance criteria. Additional selection criteria, namely vetoes of
dilepton invariant mass ranges, are well motivated by the removal
of backgrounds involving Z resonance production. The selection criteria
and background measurement procedures were set prior to viewing the 
data in the signal region.

This analysis shares parts of signal regions with other ATLAS super-symmetric
and exotic analyses, which are under internal review. If similar excesses are seen in these analyses,
it will provide some cross-check to the \tth\ analysis, which is done with a
different procedure for assessing fake background contributions.

This multilepton analysis will be combined with the already public $H\rightarrow b\bar{b}$ \cite{Aad:2014lma} and
$H\rightarrow\gamma\gamma$\cite{ATLAS-CONF-2014-011} analysis, which have observed(expected) 95\%
confidence limits on $\mu$ of 4.1(2.6) and 4.9(6.7), respectively. Both have seen small, non-significant excesses 
with fitted $\mu$ values of 1.7 $\pm$ 1.4 and 1.3 $+$ 2.62 $-$ 1.75, respectively. When approved,
the multilepton analysis will be combined with other analyses in fit constraining the parameters of the Higgs sector.
As it is primarily dependent on the top Yukawa coupling and Higgs coupling to W bosons, it will
have the greatest effect on the measurement of these parameters.

It is interesting to note that the CMS experiment observes a similar excess in their multilepton search
for \tth production. Their observed(expected) 95\% limit on $\mu$ is 6.6(2.4) and their fitted value of $\mu$
is 3.7 $+$ 1.6 $-$ 1.4 \cite{CMS-PAS-HIG-13-020}.

Observing \tth\ production is really a task for the second LHC run. The increased luminosity and collision
energy (13 TeV) will ensure that the \tth\ process is observed with that run's dataset \cite{Dawson:2013bba}. The
overall cross-section of \tth production will increase by a factor of roughly 4, due to accessing
more of the gluon PDFs in the collisions. The cross-sections of most of the backgrounds (\ttV\ and top fakes) will also increase by this factor,
but the increase in the signal cross-section is more important: the second run dataset will have
twice the sensitivity to \tth\ per amount of collected data. While
there will likely not be any conclusive resolution of the excesses found in the multilepton signal regions
with the first run dataset, the second run dataset will surely shed light on this issue.  

