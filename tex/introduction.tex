%% \chapter[htoc-titlei][hhead-titlei]{htitlei}
%% -----------------------------------------------------------------------------
\chapter[Introduction][Introduction]{Introduction}

The discovery of the Higgs boson at the Large Hadron Collider (LHC) experiments has opened up a new paradigm of research into the
Standard Model of particle physics.
This thesis primarily documents a search for the production of Higgs boson in association with top quarks (\tth) 
in multi-lepton final states.  Searching for this production mode of the Higgs is an important
step toward a precise measurement of the top Yukawa coupling, because it accesses this coupling via diagrams
that do not contain loops. Comparison of this coupling with the already well-measured top quark mass provides
a direct test of a fundamental provision of the Higgs mechanism: that it gives mass to the fermions. 

The analysis uses the 2012 ATLAS experiment's dataset of proton-proton collisions at 
a center-of-mass energy of 8 TeV provided by the LHC. The statistics available do not allow for an observation of the \tth\ process
at the Standard Model production cross-section, and the results of the search are interpreted as
a 95\% exclusion on the production rate. The results will provide some of strictest constraints on the rate to date and
establish a program for future analyses on larger datasets that will eventually observe this production mode. 

Chapter~\ref{chapter:theory} provides theoretical background and motivation for the study of this particular
Higgs production mode and Chapter~\ref{chapter:lhc} provides a basic review of the experimental apparatus,
the LHC and ATLAS. Chapter~\ref{chapter:electron} is a brief diversion
from the main text to elaborate on the techniques used to identify electrons and measure their identification
efficiency. 

Chapters~\ref{chapter:analysis}-\ref{chapter:results} are the main text, which discuss the full analysis
procedure for the search and the final measurement. The results of the analysis have been approved by the
in the ATLAS collaboration and eventually will be documented for publication. They will eventually
be combined with other Higgs searches to set limits on Higgs couplings to other SM particles,
particularly the top quark. 
